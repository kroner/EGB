  %  File:  EGB_algorithm.tex

\documentclass{amsart}
\usepackage{amssymb}
 
\newtheorem{theorem}{Theorem}[section]
\newtheorem{lemma}[theorem]{Lemma}
\newtheorem{proposition}[theorem]{Proposition}
\newtheorem{corollary}[theorem]{Corollary}
\newtheorem{question}[theorem]{Question}\newtheorem{conjecture}[theorem]{Conjecture}

\theoremstyle{definition}
\newtheorem{definition}[theorem]{Definition}
\newtheorem{example}[theorem]{Example}
\newtheorem{problem}[theorem]{Problem}
\newtheorem{xca}[theorem]{Exercise}
\newtheorem{algorithm}[theorem]{Algorithm}

\theoremstyle{remark}
\newtheorem{remark}[theorem]{Remark}
\newtheorem{criterion}[theorem]{Criterion}

\numberwithin{equation}{section}

%    Absolute value notation
\newcommand{\abs}[1]{\lvert#1\rvert}

%    Blank box placeholder for figures (to avoid requiring any
%    particular graphics capabilities for printing this document).
\newcommand{\blankbox}[2]{%
  \parbox{\columnwidth}{\centering
%    Set 
fboxsep to 0 so that the actual size of the box will match the
%    given measurements more closely.
    \setlength{\fboxsep}{0pt}%
    
\fbox{\raisebox{0pt}[#2]{\hspace{#1}}}%
  }%
}

\newcommand{\floor}[1]{\lfloor {#1} \rfloor}
\newcommand{\ceil}[1]{\lceil {#1} \rceil}
\newcommand{\binomial}[2]{\left( 
\begin{array}{c} {#1} \\
                        {#2} \end{array} \right)}
\def\la{\langle}
\def\ra{\rangle}

\newcommand{\lm}{\operatorname{lm}}
\newcommand{\lc}{\operatorname{lc}}
\newcommand{\lt}{\operatorname{lt}}


\newcommand{\rank}{\textit{\rm rank}}
\newcommand{\<}{\langle}
\renewcommand{\>}{\rangle}
\newcommand{\ideal}[1]{\langle #1 \rangle}
\newcommand{\LT}{\operatorname{in}_>}


\DeclareSymbolFont{AMSb}{U}{msb}{m}{n}
\DeclareMathSymbol{\F}{\mathbin}{AMSb}{"46}
\DeclareMathSymbol{\N}{\mathbin}{AMSb}{"4E}
\DeclareMathSymbol{\Z}{\mathbin}{AMSb}{"5A}
\DeclareMathSymbol{\R}{\mathbin}{AMSb}{"52}
\DeclareMathSymbol{\C}{\mathbin}{AMSb}{"43}


\begin{document} \title[Algorithms for Symmetric Gr\"obner Bases]
{An Algorithm for Finding \\  Gr\"obner Bases in Infinite \\ Dimensional Polynomial Rings}

 %   Information for first author 
\author{Christopher J. Hillar}
\address{Redwood Center for Theoretical Neuroscience, University of California, Berkeley}
\email{chillar@msri.org}

\author{Robert Krone}
\address{Georgia Tech University, Atlanta, GA}
\email{krone@math.gatech.edu}

\author{Anton Leykin}
\address{Georgia Tech University, Atlanta, GA}
\email{anton.leykin@gmail.com}

%\thanks{The work of the first author is supported under a National Science Foundation Graduate Research Fellowship.} 

\subjclass{13E05, 13E15, 20B30, 06A07}%

% \keywords{Invariant ideal, well-quasi-ordering, symmetric group, Gr\"obner basis, generating sets}


% \keywords{}

% ----------------------------------------------------------------
\begin{abstract}
We give an explicit algorithm to find Gr\"obner bases for symmetric
ideals in infinite dimensional polynomials rings.  This allows for symbolic computation
in a new class of rings.
\end{abstract} 

\maketitle 

\section{Introduction}
Let $X = \{x_1,x_2,\ldots\}$ be an infinite collection of
indeterminates, indexed by the positive integers, and let ${\mathfrak S}_{\infty}$ be the group of
permutations of $X$.  For a positive integer $N$, we will also let 
${\mathfrak S}_N$ denote the set of permutations of $\{1,\ldots,N\}$.
Fix a field $K$ and let $R = K[X]$ be the polynomial ring in the indeterminates $X$.  The group
${\mathfrak S}_{\infty}$ acts naturally on $R$: if $\sigma \in {\mathfrak
  S}_{\infty}$ and $f\in K[x_1,\dots,x_n]$, then 
\begin{equation}\label{groupaction}
\sigma f(x_1,\ldots,x_n) = f( x_{\sigma1},\dots,  x_{\sigma n})\in R.
\end{equation}
We let $R[{\mathfrak S}_{\infty}]$ be the (left) group ring of ${\mathfrak S}_{\infty}$ over $R$ 
with multiplication given by $f\sigma\cdot g\tau = fg(\sigma\tau)$ for $f,g\in R$ and
$\sigma,\tau\in {\mathfrak S}_{\infty}$, and extended by linearity.
The action (\ref{groupaction}) naturally gives $R$ the structure of a (left) module over the
ring $R[{\mathfrak S}_{\infty}]$.
An ideal $I \subseteq R$ is called \textit{invariant under ${\mathfrak S}_{\infty}$}
(or simply \textit{invariant}) if \[ {\mathfrak S}_{\infty}I := \{\sigma f
: \sigma \in {\mathfrak S}_{\infty}, \ f \in I\} \subseteq I.\] 
Invariant ideals are then simply the $R[{\mathfrak S}_{\infty}]$-submodules of
$R$.  

The following says that while ideals of $R$
are too big in general, those with extra structure have finite presentations.

\begin{theorem}\label{onevarfinitegenthm}
Every invariant ideal of $R$ is finitely generated as an $R[ {\mathfrak S}_{\infty}]$-module.  
In other words, $R$ is a Noetherian $R[{\mathfrak S}_{\infty}]$-module.
\end{theorem}


For the purposes of this work, we will use the following notation.
Let $B$ be a ring and let $G$ be a subset of a $B$-module $M$.  Then
$\<f: f \in G \>_{B}$ will denote the $B$-submodule of $M$ generated
by the elements of $G$.


\begin{example}
$I = \<x_1,x_2,\ldots\>_R$ is an invariant ideal of $R$.  Written as 
a module over the group ring $R[{\mathfrak S}_{\infty}]$, it has the
compact presentation $I= \<x_1\>_{R[{\mathfrak S}_{\infty}]}$.
\end{example}

\begin{theorem}
Let $G$ be a Gr\"obner basis for an invariant ideal $I$.  Then $f \in I$ 
if and only if $f$ has normal form $0$ with respect to $G$.
\end{theorem}

\begin{example}\label{trunccounterex}
Let $I = \<x_1 + x_2, x_1 x_2\>_{R[{\mathfrak S}_{\infty}]}$.  Then, a
Gr\"obner basis for $I$ is given by $G = \{x_1\}$.  It is important to note that we may not 
simply restrict consideration to $K[x_1,x_2]$ to produce this result since
%\[  \<x_1 + x_2, x_1 x_2\>_{R[{\mathfrak S}_{\infty}]} = \<x_1\>_{R[{\mathfrak S}_{\infty}]}\]
%but 
\[  \<x_1 + x_2, x_1 x_2\>_{R[{\mathfrak S}_{2}]} \neq  \<x_1\>_{R[{\mathfrak S}_{2}]}.\]
\end{example}

\begin{example}
The ideal $I = \<x_1^3 x_3 + x_1^2 x_2^3, 
x_2^2 x_3^2 - x_2^2 x_1 + x_1 x_3^2\>_{R[{\mathfrak S}_{\infty}]}$ has a 
Gr\"obner basis given by:
\[ G = {\mathfrak S}_{3} \cdot \{x_3 x_2 x_1^2, x_3^2 x_1 + x_2^4 x_1 - x_2^2 x_1, 
x_3 x_1^3, x_2 x_1^4, x_2^2 x_1^2\}.\]
Once $G$ is found, testing whether a polynomial $f$ is in $I$ is computationally fast.
\qed
\end{example}

The normal form reduction we are talking about here is a modification of the
standard notion in polynomial theory and Gr\"obner bases;  
we describe it in more detail in Section \ref{gbinfinite}.
Unfortunately, the techniques used to
prove finiteness in \cite{AH} are nonconstructive and therefore do not
give methods for computing Gr\"obner bases in $R$.
Our main result is an algorithm for finding these bases.

\begin{theorem}
Let $I = \<f_1, \ldots, f_n\>_{R[{\mathfrak S}_{\infty}]}$ be an invariant ideal of $R$.
There exists an effective algorithm to compute a finite minimal Gr\"obner basis for $I$.
\end{theorem}

\begin{corollary}
There exists an effective algorithm to solve the ideal membership problem
for symmetric ideals in the infinite dimensional ring $K[x_1,x_2,\ldots]$.
\end{corollary}

Let $G$ denote the monoid of strictly increasing functions $\pi : \N \to \N$.  $G$ has a natural action on the variables of $R$ with $\pi$ mapping $x_i$ to $x_{\pi(i)}$.  This defines an action of $G$ on $R$ and note that this action respects the term order defined on $R$.  For any two monomials $m,n$, $m < n$ implies $\pi(m) < \pi(n)$ for all $\pi \in G$.  In particular for any polynomial $g \in R$, $\LT(\pi g) = \pi \LT(g)$.  We can define the $G$-invariant ideals as ideals $I \subset R$ with $G I \subset I$.  Equivalently they are the $R[G]$-submodules of $R$.
\begin{proposition}
 $R$ is a Noetherian $R[G]$-module.
\end{proposition}

\begin{proposition}
 If $I$ is a ${\mathfrak S}_\infty$-invariant ideal, then $I$ is also $G$-invariant.
\end{proposition}
\begin{proof}
 For any $f \in I$ and $\pi \in G$, $\pi f$ contains only a finite number of variables in $R$.  Let $n$ be the index of the largest variable occurring in $\pi f$.  Then there exists $\sigma \in {\mathfrak S}_n$ such that $\pi f = \sigma f$.  Because $I$ is ${\mathfrak S}_\infty$-invariant, $\sigma f \in I$ so $\pi f \in I$.
\end{proof}

For any set $F \subset R$ then $\ideal{F}_{R[G]} \subset \ideal{F}_{R[{\mathfrak S}_\infty]}$.

Given a finite set $F$ that generates a ${\mathfrak S}_\infty$-invariant ideal $I$ as a $R[{\mathfrak S}_\infty]$-module, we can easily construct a finite set that generates $I$ as a $R[G]$-module as follows.
\begin{proposition}
 Let $F$ be a finite set of polynomials and let $n$ be the index of the largest variable occurring in $F$.  Then $\ideal{F}_{R[{\mathfrak S}_\infty]} = \ideal{{\mathfrak S}_n F}_{R[G]}$.
\end{proposition}
\begin{proof}
 Clearly $\ideal{{\mathfrak S}_n F}_{R[G]} \subset \ideal{F}_{R[{\mathfrak S}_\infty]}$.

 To prove the other containment, it's enough to show for any $f \in F$ and $\sigma \in {\mathfrak S}_\infty$, that $\sigma f \in \ideal{{\mathfrak S}_n F}_{R[G]}$ since the elements of this form generate $\ideal{F}_{R[{\mathfrak S}_\infty]}$ as an ideal.  Let $\tau \in {\mathfrak S}_n$ be the permutation that puts the integers 1 to $n$ in the same order as $\sigma$ does, i.e. $\tau(i) < \tau(j)$ exactly when $\sigma(i) < \sigma(j)$ for all $i,j \in \{1,\ldots,n\}$.  Then there is some monoid element $\pi \in G$ such that $\sigma(i) = \pi \circ \tau(i)$ for all $i \in \{1,\ldots,n\}$.  Therefore $\sigma f = \pi (\tau f) \in \ideal{{\mathfrak S}_n F}_{R[G]}$.
\end{proof}

Given a ${\mathfrak S}_\infty$-invariant ideal $I = \ideal{F}_{{\mathfrak S}_\infty}$, we will use the set ${\mathfrak S}_n F$ which generates $I$ as a $R[G]$-module to find a $G$-Gr\"obner basis of $I$ (defined below) and from that obtain a ${\mathfrak S}_\infty$-Gr\"obner basis.
\begin{definition}
 A $G$-Gr\"obner basis of a $G$-invariant ideal $I$ is a finite set $B \subset I$ such that for any $f \in I$, there exists $g \in B$ with $\pi \LT{g}$ dividing $\LT{f}$ for some $\pi \in G$.
\end{definition}

Note that if $I$ is also ${\mathfrak S}_\infty$-invariant, then a $G$-Gr\"obner basis of $I$ is also a ${\mathfrak S}_\infty$-Gr\"obner basis.  Given $f \in I$, $g \in B$ and $\pi \in G$ with $\pi \LT{g}|\LT{f}$, $\pi$ can also be expressed as a permutation $\sigma \in {\mathfrak S}_\infty$ and $\sigma$ witnesses $\LT{g} \preceq \LT{f}$ since it only increases the indices of the variables in $\LT{g}$ and maintains their relative order.

Using the work of Brouwer and Draisma, there is an algorithm for finding a $G$-Gr\"obner basis of $I$ provided that the action of $G$ satisfies certain criteria, which we will show it does.
\begin{proposition}[Satisfaction of Criterion EGB4]
 For all $f,h \in R$, the set $Gf \times Gh$ is the union of a finite number of $G$-orbits (where $G$ acts diagonally on $R \times R$), and generators of these orbits can be computed effectively.
\end{proposition}
\begin{proof}
 Let $n$ be the index of the largest variable occurring in $f$ and $g$.  Let $H$ be the set of strictly increasing maps $\tau:[n] \to [2n]$, which is finite.  A given pair $(\pi_1 f, \pi_2 g) \in Gf \times Gh$ is determined by the images $\pi_1([n])$ and $\pi_2([n])$.  Let $S = \pi_1([n]) \cup \pi_2([n])$ and denote the elements in this set $s_1 < \cdots < s_k$.  Note that $|S| = k \leq 2n$.  Let $\sigma:[k] \to S$ be defined by $i \mapsto s_i$.  Let $\tau_i:[n] \to [k]$ be the map $\sigma^{-1} \circ \pi_i$.  Then $(\pi_1 f, \pi_2 g) = \sigma(\tau_1 f,\tau_2 g)$ and $\sigma$ can be extended to an element of $G$ so the pairs $(\tau_1 f,\tau_2 g)$ include all $G$ orbits of $Gf \times Gh$.  The number of orbits is finite since $H$ is finite, and they can easily be computed by enumerating over all possible pairs $\tau_1,\tau_2 \in H$.
\end{proof}

Although $|H\times H| = \binomial{2n}{n}^2$, we can use a considerably smaller subset of pairs of maps $\tau_1$ and $\tau_2$.  In particular, given any $\pi_1, \pi_2 \in G$, $\tau_1$ and $\tau_2$ are determined only by the two subsets $S_i = \{j \in [n]: \pi_i(j) \in \pi_1([n]) \cap \pi_2([n])\} \subset [n]$.  Note that $|S_1| = |S_2|$, so the number of possible pairs of subsets is
\[ \sum_{k=0}^n \binom{n}{k}^2 = \binom{2n}{n} = O(4^n/\sqrt{n}). \]


\section{Algorithms}

We postpone the proof of correctness of the algorithms above until Section \ref{proofcorrect}

\section{Examples and Problems}\label{algexamples}

Here we list some examples of our algorithm.\footnote{Code that performs the calculations in this section can be found at ??.}

\begin{example}
Consider $F = \{x_1+x_2, x_1 x_2\}$ from the introduction.  One step of the Algorithm gives $F' = \{x_1+x_2, x_1^2\}$.
The next two iterations produce $\{x_1\}$ and thus the algorithm returns with this as its answer.  [*** Check this with M2 code ***]
\end{example}

If the input polynomials $F$ are linear, then a reduced Gr\"obner basis in the classical sense corresponds to a Gaussian elimination calculation.

\begin{problem}
What is a $G$-basis for a finite set of linear polynomials $F$?
\end{problem}

In experiments, we find that generic polynomials give rise to simple invariant ideals.

\begin{problem}
What is the $G$-basis of a principal invariant ideal $I = \langle f \rangle$, where $f$ is a generic polynomial of degree $d$?
\end{problem}

\begin{problem}
Consider the matrix:
\[ X =  \left[\begin{array}{cccc}x_1 & x_2 & x_3 & \cdots \\x_2 & x_3 & x_4 & \cdots \\x_3 & x_4 & x_5 & \cdots\end{array}\right].\]
What is the $G$-basis for all $3 \times 3$ minors of $X$?
\end{problem}



\section{Proof of Correctness}\label{proofcorrect}

Here we prove that our algorithm terminates and 
produces a Gr\"obner basis for an ideal $I$.




% ----------------------------------------------------------------
\begin{thebibliography}{99}

%\bibitem{AH}
%M. Aschenbrenner and C. Hillar, \emph{Finite generation of symmetric ideals}, 
%Trans. Amer. Math. Soc., to appear.

%\bibitem{BB1} B. Buchberger, \emph{An algorithm for finding a basis
%for the residue class ring of a zero-dimensional polynomial
%ideal}, PhD Thesis, University of Innsbruck,
%Institute for Mathematics (1965).

%\bibitem{BB2} B. Buchberger, \emph{An algorithmic criterion for the
%solvability of algebraic systems of equations},
%Aequat. Math. \textbf{4} (1970), 374--383.

\bibitem{Cox2} D. Cox, J. Little, D. O'Shea, \emph{Using algebraic geometry},
Springer, New York, 1998.

%\bibitem{HW} 
%C. Hillar and T. Windfeldt, \emph{Minimal generators for symmetric ideals}, preprint.

%\bibitem{Ruch1} A. Mead, E. Ruch, A. Sch\"onhofer, \emph{Theory of
%    chirality functions, generalized for molecules with chiral
%    ligands}.  Theor. Chim.  Acta \textbf{29} (1973), 269--304.
%
%\bibitem{Ruch2} E. Ruch, A. Sch\"onhofer, \emph{Theorie der
%    Chiralit\"atsfunktionen}, Theor. Chim. Acta \textbf{19} (1970),
%  225--287.
%
%\bibitem{Ruch3} E. Ruch, A. Sch\"onhofer, I. Ugi, \emph{Die
%    Vandermondesche Determinante als N\"aherungsansatz f\"ur eine
%    Chiralit\"atsbeobachtung, ihre Verwendung in der Stereochemie und
%    zur Berechnung der optischen Aktivit\"at}, Theor. Chim.  Acta
%  \textbf{7} (1967), 420--432.

%\bibitem{JS}
%J. Schicho, private communication, 2006.



\end{thebibliography}

\end{document}
% ----------------------------------------------------------------

